% ======================================================
% Configuration Manual – N-of-1 ADHD + Bipolar Disorder
% Practicum Part 2 – MSc Artificial Intelligence for Business
% National College of Ireland – Rodrigo Marques Teixeira
% ======================================================

\documentclass[12pt,a4paper]{article}

% ---------- Packages ----------
\usepackage[utf8]{inputenc}
\usepackage[T1]{fontenc}
\usepackage{graphicx}
\usepackage{geometry}
\usepackage{hyperref}
\usepackage{titlesec}
\usepackage{setspace}
\usepackage{fancyhdr}
\usepackage{booktabs}
\usepackage{longtable}
\usepackage{float}
\usepackage{caption}
\usepackage{pgfplots}

% ---------- Layout ----------
\geometry{margin=2.5cm}
\setstretch{1.3}
\pagestyle{fancy}
\fancyhf{}
\rhead{Configuration Manual – Practicum Part 2}
\lhead{Rodrigo Marques Teixeira}
\cfoot{\thepage}

% ---------- Title ----------
\title{\textbf{Configuration Manual (Full Version)}\\N-of-1 Study – ADHD + Bipolar Disorder\\\vspace{0.4cm}\large Practicum Part 2 – MSc in Artificial Intelligence for Business}
\author{\textbf{Rodrigo Marques Teixeira} \\ National College of Ireland \\ Supervisor: Dr. Agatha Mattos}
\date{October 2025}

% ---------- Document ----------
\begin{document}
\maketitle
\tableofcontents
\newpage

% ======================================================
% 1. Introduction
% ======================================================
\section{Introduction}
This Configuration Manual describes the technical and ethical setup of the Practicum Part 2 project, focusing on the reproducible data pipeline and modeling workflow for the N-of-1 longitudinal study on comorbidity ADHD and Bipolar Disorder. The document complements the brief version submitted earlier and extends it with detailed configuration, architecture, and reproducibility notes.

% ======================================================
% 2. System Architecture
% ======================================================
\section{System Architecture}
This section outlines the architecture of the project, from raw data ingestion to explainability outputs. Include or reference one system diagram (e.g., ETL → Modeling → Explainability).\\
\textbf{Main components:}
\begin{itemize}
    \item Data sources: Apple Health, Amazfit GTR4, Helio Ring, EMA.
    \item ETL pipeline: Python script with z-score normalization, segmentation (S1–S6), and NaN handling.
    \item Modeling: Kaggle notebooks executing baselines and LSTM.
    \item Explainability: SHAP drift detection and feature attribution.
\end{itemize}

% ======================================================
% 3. Environment Setup
% ======================================================
\section{Environment Setup}
\subsection{Requirements}
The environment is Python-based (version 3.10+) and relies on open-source libraries. The complete dependency list is provided in \texttt{requirements.txt}.

\subsection{Execution}
\begin{verbatim}
cd etl
pip install -r requirements.txt
python etl_pipeline.py
\end{verbatim}
For modeling, notebooks can be executed in Kaggle (GPU T4) or locally via JupyterLab.

% ======================================================
% 4. Data Management and Preprocessing
% ======================================================
\section{Data Management and Preprocessing}
Describe the segmentation (S1–S6), feature normalization, and fallback methods. Reference \texttt{etl_qc_summary.csv} for quality metrics. Detail how missing data are handled and aggregated daily.

% ======================================================
% 5. Modeling Framework
% ======================================================
\section{Modeling Framework}
Summarize the notebook workflows:\\
\textbf{01\_feature\_engineering.ipynb}: feature creation and aggregation.\\
\textbf{02\_model\_training.ipynb}: baselines (Naïve, Logistic Regression) and LSTM architectures.\\
\textbf{03\_shap\_analysis.ipynb}: explainability with SHAP and drift detection.\\
\textbf{04\_rule\_based\_baseline.ipynb}: heuristic comparison model.

Include mention of 6-fold temporal cross-validation, evaluation metrics (F1, AUROC, Balanced ACC, Cohen’s kappa), and export of best model as \texttt{best\_model.tflite}.

% ======================================================
% 6. Ethics and Governance
% ======================================================
\section{Ethics and Governance}
Summarize the ethical scope and compliance procedures:\\
\textbf{Phase 1:} self-data collection.\\
\textbf{Phase 2:} expansion to relatives/friends with informed consent and anonymisation.\\
Refer to the \textit{Ethics Submission Plan} and Consent documentation stored under \texttt{/docs}.

% ======================================================
% 7. Reproducibility and Version Control
% ======================================================
\section{Reproducibility and Version Control}
Describe GitHub repository structure, tagging conventions (v2.0–v2.3), and dataset version tracking. Mention that the repository contains only anonymised or synthetic examples.

% ======================================================
% 8. Future Work and Extensions
% ======================================================
\section{Future Work and Extensions}
Outline potential expansions such as:\\
\begin{itemize}
    \item S7–S9 data collection phases with new participants.
    \item Automated drift detection and retraining.
    \item Integration of emotion data from Helio Ring (via Zepp Cloud API).
    \item Validation of heuristic labels with clinical psychology collaborators.
\end{itemize}

% ======================================================
% Appendix
% ======================================================
\appendix
\section{Appendix}
Include diagrams, configuration tables, or excerpts of key scripts (e.g., ETL pipeline pseudocode).

\end{document}
